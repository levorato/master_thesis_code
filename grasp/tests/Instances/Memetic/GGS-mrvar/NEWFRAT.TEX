\documentclass[12pt,a4paper]{article}
\usepackage[dvips]{epsfig}
\usepackage[dvips]{graphics}
\usepackage{sloven2e}
\usepackage{times}
\usepackage{drawnet}
\usepackage{csz}

\newcommand{\htp}[1]{\footnotesize\sf{#1}\normalsize\rm}
\newcommand{\ttx}[1]{\small\sf{#1}\normalsize\rm}
\parindent 0pt
\parskip 6 pt
% *** floats control

\begin{document}

\pagestyle{empty}
{\bf Newcomb fraternity} (bratstvo, zdru"zenje) (UCINET)

Podatki iz leta 1956, 17 "studentov, ki se od prej ne poznajo.
"Studenti postanejo sostanovalci v "studentskem naselju Univerze v 
Michiganu.

Vsak od 17 "studentov je razdelil ocene $1 \ldots 16$ 
ostalim "studentom. Ocene so rangi glede na relacijo 'imeti rad'.
Izbrani "student je dal drugemu "studentu rang 1, 
"ce ima tega "studenta najraj"si, rang 16,
"ce ima tega "studenta  najmanj rad\ldots

Ocenjevanje se je ponovilo v 15 tednih (datoteke newc0 do newc15),
manjkajo pa podatki za teden 9.
Omre"zje v prvem tednu:
\begin{small}
\begin{verbatim}
   A  B  C  D  E  F  G  H  I  J  K  L  M  N  O  P  Q
A  0  7 12 11 10  4 13 14 15 16  3  9  1  5  8  6  2
B  8  0 16  1 11 12  2 14 10 13 15  6  7  9  5  3  4
C 13 10  0  7  8 11  9 15  6  5  2  1 16 12  4 14  3
D 13  1 15  0 14  4  3 16 12  7  6  9  8 11 10  5  2
E 14 10 11  7  0 16 12  4  5  6  2  3 13 15  8  9  1
F  7 13 11  3 15  0 10  2  4 16 14  5  1 12  9  8  6
G 15  4 11  3 16  8  0  6  9 10  5  2 14 12 13  7  1
H  9  8 16  7 10  1 14  0 11  3  2  5  4 15 12 13  6
I  6 16  8 14 13 11  4 15  0  7  1  2  9  5 12 10  3
J  2 16  9 14 11  4  3 10  7  0 15  8 12 13  1  6  5
K 12  7  4  8  6 14  9 16  3 13  0  2 10 15 11  5  1
L 15 11  2  6  5 14  7 13 10  4  3  0 16  8  9 12  1
M  1 15 16  7  4  2 12 14 13  8  6 11  0 10  3  9  5
N 14  5  8  6 13  9  2 16  1  3 12  7 15  0  4 11 10
O 16  9  4  8  1 13 11 12  6  2  3  5 10 15  0 14  7
P  8 11 15  3 13 16 14 12  1  9  2  6 10  7  5  0  4
Q  9 15 10  2  4 11  5 12  3  7  8  1  6 16 14 13  0
\end{verbatim}
\end{small}
Z rekodiranji lahko iz teh omre"zij zgradimo ve"c ozna"cenih grafov:

\begin{enumerate}
\item rekodiranje (datoteke New1*.net):\\
 $1, 2, 3 \to 1$ (ima rad)\\
 $14, 15, 16 \to -1$ (ne mara)\\
 $4, 5, 6, 7, 8, 9, 10, 11, 12, 13 \to 0$.
\item rekodiranje (datoteke New2*.net):\\
 $1, 2, 3 \to 1$ (ima rad)\\
 $15, 16 \to -1$ (ne mara)\\ 
 $4, 5, 6, 7, 8, 9, 10, 11, 12, 13, 14 \to 0$. 
\item rekodiranje (datoteke New3*.net):\\
 $1, 2, 3, 4 \to 1$ (ima rad)\\
 $14, 15, 16 \to -1$ (ne mara)\\
 $5, 6, 7, 8, 9, 10, 11, 12, 13 \to 0$.
 
 
\end{enumerate}   

Newcomb, T. (1962): \\
{\em The acquaintance process.} New York: Holt. Reinhart \& Winston.

\end{document}