\documentclass[12pt,a4paper]{article}
\usepackage[dvips]{epsfig}
\usepackage[dvips]{graphics}
\usepackage{sloven2e}
\usepackage{times}
\usepackage{drawnet}
\usepackage{csz}

\newcommand{\htp}[1]{\footnotesize\sf{#1}\normalsize\rm}
\newcommand{\ttx}[1]{\small\sf{#1}\normalsize\rm}
\parindent 0pt
\parskip 6 pt
% *** floats control

\begin{document}
\pagestyle{empty}
{\bf Sampson Monastery}

Sampson je preu"ceval odnose med 18 menihi v samostanu 
New England. Izmeril je ve"c relacij med njimi

\begin{itemize}
\item prijateljstvo (affect)
\item spo"stovanje (esteem)
\item vplivnost (influence)
\item odobravanje (sanction)
\end{itemize}

Za relacijo prijateljstva imamo podatke za tri "casovne to"cke 
$T_2$, $T_3$ in $T_4$, za ostale tri relacije pa samo v "casovni to"cki $T_4$.

Sampson je relacije med menihi podal v obliki
vrednostnih ozna"cenih grafov. Vsak menih je izbral $3$ druge, s katerimi
se najbolje razume, in jih ocenil z vrednostmi $1$, $2$ ali $3$, 
kjer $3$ pomeni najmo"cnej"se, $1$ pa naj"sibkej"se prijateljstvo.
Prav tako je vsak izbral $3$ menihe, s katerimi se najslab"se razume 
in jih ocenil z vrednostmi $-1$, $-2$ in $-3$, kjer spet $-3$ 
pomeni najmo"cnej"se, $-1$ pa naj"sibkej"se sovra"stvo. 

Relacija prijateljstva med menihi v "casu $T_2$:
{\small
\begin{center}
\begin{tabular}{|r|l|@{\ }*{18}{@{  }r @{  }}|}
\hline
"St.& Menih     &~1& ~2& ~3& ~4& ~5& ~6& ~7& ~8& ~9&10&11&12&13&14&15&16&17&18 \\ \hline
1 &JohnBosco & 0& 0& 2&0& 3&-2&-1& 0& 0&-3& 0& 0& 0& 1& 0& 0& 0& 0 \\
2 &Gregory    & 3& 0& 0& 0& 0& 0& 2&0& 0&-1& 0& 0&-3& 1& 0& 0&-2&0 \\
3 &Basil      & 2&3& 0&-1& 0& 0& 0&-3&-2&0& 0& 0& 0& 0& 0& 0& 1& 0 \\
4 &Peter      & 0& 0&-2&0& 3& 1&-3& 0& 0& 2&0& 0& 0& 0& 0& 0& 0&-1 \\
5 &Bonavent. & 0& 0& 0& 3& 0& 0& 0& 0& 0& 0& 2&0& 1& 0& 0& 0& 0& 0 \\
6 &Berthold   & 1& 0& 0& 3& 0& 0&-3&-1& 2&0& 0&-2&0& 0& 0& 0& 0& 0 \\
7 &Mark       & 0& 2&0&-3&-1&-2&0& 1& 0& 0& 0& 0& 0& 0& 0& 3& 0& 0 \\
8 &Victor     & 3& 2&-3& 0& 0& 0& 0& 0& 1& 0& 0& 0& 0&-2&0& 0&-1& 0 \\
9 &Ambrose    & 0& 0&-3& 0& 2&0& 0& 3& 0& 0& 0& 0& 0& 0& 0& 1&-2&-1 \\
10 &Romuald   & 0& 0& 0& 3& 0& 0& 0& 1& 0& 0& 0& 0& 0& 2&0& 0& 0& 0 \\
11 &Louis     & 0& 0&-1& 0& 3& 0& 0& 1& 0& 0& 0& 0&-3& 2&0& 0&-2&0 \\
12 &Winfrid   & 3& 2&-1&-3& 0& 0& 0& 0& 0& 0&-2&0& 0& 1& 0& 0& 0& 0 \\
13 &Amand     & 0&-3& 0& 0& 2&-2&1& 0& 0& 0& 0&-1& 0& 0& 0& 0& 0& 3 \\
14 &Hugh      & 3& 0& 0& 0& 0& 0& 0&-2&0& 0& 1& 2&-3& 0& 2&0&-1& 0 \\
15 &Boniface  & 3& 2&-2&0& 0& 0& 0& 0& 0& 0& 0& 0&-3& 1& 0& 0&-1&-1 \\
16 &Albert    & 1& 2&0& 0& 0& 0& 3& 0& 0& 0& 0& 0&-1& 0& 0& 0&-3&-2 \\
17 &Elias     & 0& 0& 3&-3&-2&0& 0& 0& 0& 0&-1& 0& 2&0& 0& 0& 0& 1 \\
18 &Simplic.& 2&3& 0&-3& 0&-2&1& 0& 0& 0& 0& 0& 0& 0& 0&-1& 0& 0 \\ \hline
\end{tabular} 
\end{center}
}

Sampson, S (1969): \\
{\em Crysis in a cloister.} Unpublished doctoral dissertation. Cornell University.
\end{document}

